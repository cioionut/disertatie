\chapter{Introducere}

Dincolo de utilitatea pragmatică de a transmite informații, limbajul intensifică interacțiunea și menține legăturile între oameni fapt surprins extraordinar de un proverb românesc care spune: \textit{"Vorba dulce mult aduce"}. Așa cum menționează și Jurafski în cartea sa Speach Processing, întreaga literatură abundă în creații ale omului care capătă puterea de a vorbi. Poate că longevitatea acestui curent ne face să il abordăm cu mai multă seriozitate, trecând de le filosofie și încercarea de a înțelege comunicarea la soluții matematice care ne permit cu adevărat să creăm un astfel de agent conversațional.

Mergând pe acest drum al construirii unui sistem de dialog în literatura de specialitate se disting două ramuri: cea a dezvoltării asistenților digitali orientați pe îndeplinirea de sarcini, în general conversația nu este menită dă dureze mult, iar rolul agentului este de a capta informații din conversația cu utilizatorul și de a-l ajuta să ducă la bun sfârșit o operație. În cea de-a doua direcție este vorba despre chatbots, acest termen face referire la un agent orientat spre discuție la nivel general, servind mai de grabă unui scop de conversațional subiectiv decât unuia obiectiv soluțional. În această situație dialogul poate dura mai mult, existând și aplicații practice cum ar fi cea de a testa anumite teorii psihologice, cum a fost în cazul ELIZA \cite{weizenbaum}

Privind comunicarea ca o nevoie de bază ne ajută sa vedem mai clar de ce procesarea limbajului natural este un element esențial in drumul nostru spre cunoaștere. Datorită actualului progres in acest domeniu ne putem bucura de ușurința cu care informațiile circulă între noi, făcând realizabil acest avânt tehnologic de care nu bucurăm cu toții.

% sa spun despre abordari anterioare si referinte la ele, cum m-am gandit eu si experimentat (pe scurt)
% problematica
% contributia mea
% rezumatul capitolelor
% baseline rezultate
% sa scriu despre analiza erorilor cu exemplu, unde greseste

\section{Motivația}

Luând contact tot mai des cu mediul de cercetare, încerc să observ modul în care această comunitate reușește să aducă contribuții în societate. Un lucru care m-a făcut sa prețuiesc fiecare efort, este acela că în general o descoperire se bazează pe cercetări anterioare și ca orice gând exprimat în scopul de a descoperi, poate fi valoros și dus mai departe spre cunoaștere.

Dorința de a crea cu scopul de face viața oamenilor mai ușoară este imboldul intrinsec ce ghidează acțiunile mele, așadar evaluând cunoștințele mele, am decis sa îmi aduc contribuția într-un domeniu atât de important în drumul nostru spre o viziune clara.

\section{Descrierea problemei}

Lucrarea își propune descoperirea unui model îndeajuns de puternic încât să înțeleagă limbajul natural exprimat într-un anumit context (general sau specific unui domeniu). Și punerea în aplicare a unui administrator de dialog, suficient de simplu dar care să ducă la bun sfărșit sarcinile dorite spre a fi executate de care agentul virtual rezultat.

În viziunea mea, această chestiune reprezintă mai degrabă o nevoie decât o problemă. O nevoie ce survine în urma stilului nostru de viață dinamic și învelit în straturi de informație.

Cum limbajul natural este cel mai la îndemâna instrument de comunicare, consider ca prin intermediul său vom putea satisface nevoia unei interfețe capabile să ușureze interacțiunea dintre noi și tehnologie.

\subsection{Privire de ansamblu}

Avem nevoie de un modul de NLU și un modul de DM pentru a pune bazele unui sistem de dialog

\section{Rezumatul capitolelor}

\begin{description}
	\item[Introducere]  - 
	
	Capitolul întâi vorbește in principal despre modul în care această lucrare își propune să rezolve nevoia de interacțiune cu tehnologia, dar și despre un capitol istoric privit prin ochii unei motivații îndraznețe.
	\item
	Atunci când se rostește "progres" am în minte o  spirală a cunoștintelor care se bazează unele pe altele. Precum această imagine implementarea acestei tehnologi de dialog impune un anumit progres precedent, așa că în capitolul doi vor fi prezentate aceste instrumente care fac posibilă aceasta tehnologie.
	\item
	In capitolul al treilea vor fi explicate modelele matematice care stau în spatele modelelor decizionale.
	\item
	In partea a patra se prezintă modulul de înțelegerea limbajului natural
	\item
	Al cincelea capitol descrie modulul care ține contextul unei conversații.
	\item
	Iar in partea de final concluziile referitoare la studiul elaborat in această teză.
\end{description}


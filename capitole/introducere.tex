\chapter{Introducere}

% sa spun despre abordari anterioare si referinte la ele, cum m-am gandit eu si experimentat (pe scurt)
% problematica
% contributia mea
% rezumatul capitolelor
% baseline rezultate
% sa scriu despre analiza erorilor cu exemplu, unde greseste

Privind dincolo de utilitatea practică de a transmite informații, limbajul reprezintă principala noastră unealtă în evoluție și nu numai, el este cel care intensifică interacțiunea și menține legăturile interumane. Sunt rare momentele în care ne gândim la această capacitate de a comunica, totuși o folosim la fel de des și ne este la fel de indispensabilă precum o funcție vitală.

Înțelesul este conceptul ce are ca principală sursă limbajul, această relație scoate în evidență contrastul dintre complexitatea inferenței și simplitatea cuvintelor. Reprezentativ pentru această legătură este proverbul \textit{"Vorba dulce mult aduce"} ce surprinde în esență trăsături definitorii precum puterea cuvântului, capacitatea afectivă și în principal caracterul contemplativ al limbajului. El în sine reprezintă o formă de artă, în detaliu paradoxală și recurentă, poate din cauza faptului ca ne reprezintă ajunge să se confunde cu noi.

În cartea sa Speech and Language Proccessing \cite{speach__lang_processing}, profesorul Daniel Jurafsky observă faptul că există numeroase exemple în care omul încearcă să comunice cu creația sa. Acest comportament este surprins adesea și în literatură unde personificarea nu se oprește numai la natură. În lumina acestor observații se remarcă longevitatea și neoboseala dorinței creatoare, a cărei împlinire se concretizează în această ramură a lingvisticii matematice unde puterea de a înțelege limbajul este folosită la implementarea unor modele matematice capabile de o conversație.

%%%%%%%%%%%%%%%%%%%%%%%%%
\subsection{Progres precedent}
Mergând pe acest drum, în literatura de specialitate întâlnim două abordări care tratează construcția unui sistem de dialog.

Prima vorbește despre asistenți digitali care rezolvă diferite cerințe (taskbots). În general conversația nu este menită să dureze mult, iar rolul agentului este de a capta informații din conversația cu utilizatorul și de a-l ajuta să ducă la bun sfârșit o operație \cite{joint_online_bing, att_joint_bing}.

În cea de-a doua direcție este vorba despre agenți conversaționali (chatbots). Acest termen face referire la un agent orientat spre discuție la nivel general, servind mai degrabă unui scop subiectiv decât unuia obiectiv soluțional. În această situație dialogul poate dura mai mult, existând și aplicații practice ale acestuia cum ar fi cea de a testa anumite teorii psihologice \cite{weizenbaum}

Ghidat în mare parte de impactul asupra industriei am ales ca direcție de studiu sistemul de dialog orientat pe rezolvarea cerințelor (taskbots), unde accentul se pune în primul rând pe puterea de a simula determinismul cu care un agent uman ar răspunde. Din această cauză avem o suită de module: recunoaștere vocală (Automatic Speech Recognition - ASR sau Speech To Text - STT), înțelegerea limbajului natural (Natural Language Understanding - NLU), administrarea dialogului (Dialog Manager - DM), generarea de replici (Natural Language Generation) și într-un final conversia textului în voce (Text To Speech, TTS) toate acestea conectate într-un flux de informații ce are ca prim declanșator utilizatorul. La celălalt pol, în cercetare, întâlnim tot mai des abordări în care componente ([STT], NLU, DM, NLG, [TTS]) sunt văzute ca un singur model unde învățarea se realizează de la un capăt la altul (end-to-end)\cite{end-to-end-goal-oriented}, nemaifiind nevoie de antrenarea separată. Privind dintr-o perspectivă practică, aceste modele end-to-end duc lipsă de mulțimi de antrenare din cauza faptului că necesită o atenție mai sporită, iar în ceea ce privește antrenarea administratorului de dialog, avem nevoie de simularea unui mediu de utilizatori, cerință impusă de strategia de învățare prin recompensă (Reinforcement Learning). Totuși aceste abordări reușesc să surprindă în aplicațiile de tip conversație deschisă (chatbots) unde nu se pune accent mare pe contextul dialogului.

Revenind la sistemul de dialog orientat pe rezolvarea cerințelor, observăm că minimum viabil al acestei arhitecturi este dat de componenta care înțelege limbajul natural (NLU) și cea care ține contextul conversației (DM). Aceste două module sunt necesare și suficiente pentru a putem avea cel puțin în scris un dialog cu mașina. 

Componenta de înțelegerea limbajului natural se ocupă cu detectarea intenției și a entităților aflate într-o replică venită din partea utilizatorului. De multe ori cele două chestiuni sunt văzute separat: detecția intenției ca o problemă de clasificare, respectiv recunoașterea constituenților semantici ca o problemă de etichetare a unei secvențe.
Pentru detectarea intenției se folosesc abordări precum \cite{id_classifiers}, iar pentru recunoașterea entităților sunt folosite tehnici de etichetare ca \cite{scipy_numpyeq_labeling}.

Arhitectura de codificare-decodificare împreună cu rețelele neuronale recurente își aduc contribuția cu succes la un număr mare de probleme din procesarea limbajului natural. Rezultate promițătoare se remarcă în următoarele chestiuni: recunoașterea de entități, traducerea automată  \cite{luoung_bahdanau_maning}, recunoașterea vorbirii, generarea de limbaj, rezumarea de text. Acest tip de abordare o întâlnim tot mai des și în ipostaze hibride, unde se încearcă rezolvarea concomitentă a mai multor probleme. În esență se folosește capacitatea de a capta înțelesul unei secvențe de text cu ajutorul unui codificator iar apoi se folosesc diferite decodificatoare pentru a traduce înțelesul captat în etichete specifice.

%%%%%%%%%%%%%%%%%%%%
\section{Nevoia de a comunica}

Privind comunicarea ca o nevoie de bază și limbajul ca un prim instrument al său, vedem de ce domenii științifice precum procesarea limbajului natural și lingvistica matematică reprezintă elemente cheie ale accesului nostru la informație, mai presus de atât curiozitatea stârnită aici ne poartă de la înțeles la autocunoaștere.

Stilul nostru de viață se schimbă odată cu tehnologia, acest dinamism aduce cu el noi straturi de informații ce dau naștere unor probleme de căutare, structurare și reprezentare a informației. Dat fiind rapiditatea cu care aceste inovații intră în contact cu oamenii, nevoia de a comunica cu un astfel de sistem se dorește a fi cât mai naturală. Gradul de eficiență și acces la informație cerut de generațiile tinere este în continuă creștere, iar acest lucru sporește nevoia de înțelegere a limbajului natural și menținerea unui dialog cât mai firesc între om și mașină.

Actualele modele de a structura informația se dovedesc a fi instrumente bune la stocarea și procesarea ei, pe când reprezentarea și inferența sunt două concepte aduse în prim plan de inteligența artificială care cere noi moduri de a rezona cu informațiile stocate.

Lucrarea își propune cercetarea unui model îndeajuns de puternic încât să înțeleagă limbajul natural exprimat într-un anumit context (general sau specific unui domeniu) și punerea în aplicare a unui administrator de dialog, suficient de complex pentru a ține contextul unor sarcini bine definite.

% todo: descrierea problemei mai punctual
Înțelegerea limbajului natural este un concept care poate îngloba o mulțime de probleme precum analiza de sentiment, clasificarea de text, rezumarea și multe alte chestiuni legate de semantică. În literatura de specialitate a sistemelor de dialog, NLU ține locul componentei ce este responsabilă cu detectarea intenției vorbitorului și recunoașterea entităților, aceste grupuri de cuvinte care au o însemnătate în cadrul unui domeniu.
De exemplu: "aș dori să îmi resetez parola, numele meu de utilizator este cioionut", avem ca intenție resetarea parolei, iar ca entitate numele de utilizator.

În administratorul de dialog, aceste intenții sunt văzute ca reprezentări ale obiectivelor utilizatorului. Odată ce intenția este stabilită, pentru a duce la bun sfârșit un obiectiv s-ar putea să mai avem nevoie de anumite informații așa cum în exemplu anterior am avut nevoie de numele de utilizator, lucru ce face ca entitățile să fie văzute drept argumente ale unui obiectiv.

\subsection{Contribuția noastră}

Luând contact tot mai des cu mediul de cercetare, încerc să observ modul în care această comunitate reușește să aducă contribuții în societate. Un lucru care m-a făcut să prețuiesc fiecare efort, este acela că în general o descoperire se bazează pe cercetări anterioare și ca orice gând exprimat în scopul de a descoperi, poate fi valoros și dus mai departe. Dorința de a crea cu scopul de a face viața oamenilor mai ușoară este imboldul intrinsec ce ghidează acțiunile mele. Așadar evaluându-mi cunoștințele am decis sa îmi aduc contribuția într-un domeniu atât de important.

% todo: de adaugat motivatia pentru abordarea aleasa
Ideea de a cerceta un model pentru înțelegerea limbajului, vine în egală măsură din curiozitate și din conștientizarea nevoii. Abordarea propusă conține două module, unul pentru NLU și altul pentru DM (dialog manager). Decizia care a stat la baza acestei arhitecturi este legată în primul rând de rigoarea impusă de un administrator de dialog separat în favoarea unei soluții end-to-end ce impune mulțimi de antrenare mult mai rafinate și simularea unui mediu de utilizatori. S-a urmărit și posibilitatea de a pune în producție un astfel de sistem, dar și ușurința de a crea noi agenți doar adăugând date specifice domeniului.


% todo: contribuția adusă, cum m-am gandit eu si experimentat (pe scurt)
Pentru a înțelege limbajul natural s-a folosit o arhitectură de tip encode-decoder, unde codificarea este realizată de o rețea neuronală recurentă, iar decodificarea se realizează în două faze diferite pentru fiecare problemă. Detectarea intenției se realizează cu ajutorul unui decodificator bazat pe o rețea de tip feed forward, care primește ca intrare secvența codificată și scoate la ieșire o distribuție de probabilitate peste intențiile cunoscute. A doua fază de decodificare este realizată de o a doua rețea recurentă care la fiecare pas prezice eticheta corespunzătoare unei entități. În ajutorul acestei arhitecturi s-au experimentat mai multe modele de atenție așa cum sunt descrise și în \cite{trans_luong_manning}.

Un plus adus în puterea de generalizare îl reprezintă însumarea celor două stări ascunse (înainte și înapoi) din codificator. Spre deosebire de \cite{att_joint_bing} unde se folosește o concatenare a celor două. O altă îmbunătățire are loc atunci când eroarea de antrenare din decodificatorul de intenții nu se mai propagă în codificator. Astfel el clasificând doar pe baza reprezentărilor învățate din decodificatorul de entități și a modelului separat de atenție.

% todo: baseline? rezultate pe scurt


%%%%%%%%%%%%%%%%%%%%%%%%%%%%%%%%%%%%%%%%%%%%%%%%%%
\section{Rezumatul capitolelor}
\begin{description}
	\item[Introducere]  - 
		Capitolul întâi vorbește in principal despre modul în care această lucrare își propune să rezolve nevoia de interacțiune cu tehnologia. Este prezentată problematica, motivația, abordările anterioare și contribuția adusă domeniului.
	\item[Dialog] -
		În această parte este vorba despre încercarea noastră de a structura limbajul și de a-l pune într-un cadru științific, ușor de modelat matematic, dar fără a pierde din vedere puterea cuvintelor.
	\item[Sistem de dialog] - 
		Capitolul al treilea vorbește despre experimentele noastre și descrie abordările cu cele mai bune rezultate. Aici este prezentată mulțimea de antrenare împreună cu performanța obținută, dar și o analiză privind erorile modelului.
	\item[Noțiuni teoretice] -
		In al patrulea capitol vor fi detaliate modelele matematice care stau în spatele modelelor decizionale.
	\item[Expunera tehnologiei] - 
		Atunci când se rostește "progres" am în minte o  spirală a cunoștințelor care se bazează unele pe altele. Precum această imagine implementarea unui sistem de dialog impune un anumit progres precedent, așa că aici se vor prezenta instrumentele care fac posibilă aceasta tehnologie.
	\item[Concluzii] -
		În finalul tezei se găsesc concluziile referitoare la studiul elaborat, împreună cu observațiile privind puterea de scalabilitate a unei astfel de abordări și înțelegerea diferenței dintre abordările din industrie și cele din spațiul academic.
\end{description}


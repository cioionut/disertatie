\chapter{Dialog}

Conform definiției din limba română, dialogul este modul de expunere care prezintă succesiunea replicilor dintr-o conversație care are loc între două sau mai multe persoane.

Această lucrare își propune să dea formă înțelegerii limbajului natural dintr-o perspectivă matematică, prezentând sub forma unei soluții programabile un întreg sistem de micro servicii toate funcționând sub umbrela aceluiași scop, comunicarea.

Pe parcursul lucrării se va face referire la dialog ca o secvență de replici între un om și un calculator pentru a transmite informații.
Referitor la componentele unui dialog, se va prezenta doar o abordare bazată pe componenta \textit{verbală}, celelalte componente \textit{nonverbală} (gesturi, mimică, poziția corpului) și \textit{paraverbală} (accentul, ritmul și intensitatea vorbirii) făcând obiectul altor lucrări viitoare.


\section{Conversație naturală}

Un factor cheie într-o conversație este acela că fiecare replică dintr-un dialog este o formă de \textbf{acțiune} venită din partea vorbitorului. \cite{witt1953}


\subsection{Acte de vorbire}

\section{Sistem de dialog}

\subsection{Componente}
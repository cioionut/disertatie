\chapter{Concluzii}

Vorbind dintr-o perspectivă academică, rezultatele obținute reușesc să convingă de adevărata putere a acestei arhitecturi.
Datorită rețelelor recurente se reușește captarea contextului, fapt ce duce la o putere mare de generalizare. Capacitatea modelului de a prezice crește atunci când in joc intervine și atenția, acest strat care ajută estimatorul să se concentreze pe anumite cuvinte atunci când decide asupra unei etichete.

Un punct slab al acestei abordări din punct de vedere al comercializării este că pentru faza de antrenare este nevoie de un număr mare de exemple. Un alt factor care stă în calea scalării numărului de clienți este nevoia unui expert în domeniu care să concentreze în mulțimea de antrenare intenții și exemple relevante. Construirea unei astfel de mulțimi de antrenare necesită și implicarea programatorului (sau a unei persoanei care cunoaște cum funcționează tehnologia - seq2seq), astfel încât să se asigure că exemplele construite de expertul în domeniu, au o oarecare consistență, spre exemplu: dacă s-a dat propoziția: "vreau să imi resetez parola din aplicația Saturn" etichetat cu entitatea: app: Saturn, să existe și contra exemple în care să se specifice și celalalt sens al cuvântului "Saturn", astfel modulul de NLU să poată identifica cu succes atunci când este vorba de aplicație sau despre planetă.

Legat de componenta care ține contextul întregii conversații, este foarte ușor de urmărit întregul fir al discuției, datorită modelului bazat pe umplerea de sloturi (entități necesare în cadrul unei sarcini).
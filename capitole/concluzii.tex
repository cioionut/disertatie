\chapter{Concluzii}

Vorbind dintr-o perspectivă academică, rezultatele obținute reușesc să convingă de adevărata putere a arhitecturi de codificare-decodificare.
Datorită rețelelor recurente se reușește captarea contextului, fapt ce duce la o putere mare de generalizare. Capacitatea modelului de a prezice crește atunci când in joc intervine și atenția, acest strat care ajută estimatorul să se concentreze pe anumite cuvinte atunci când decide asupra unei etichete.

Din perspectiva comercializării unei astfel de soluții există puține impedimente, dar totuși notabile și anume: în faza de antrenare este nevoie de un număr mare de exemple etichetate. Un alt factor care stă în calea scalării numărului de clienți este nevoia unui expert în domeniu care să concentreze în mulțimea de antrenare intenții și exemple relevante. Construirea unei astfel de mulțimi de antrenare necesită și implicarea dezvoltatorului (sau a unei persoanei care cunoaște cum funcționează tehnologia), astfel încât să se asigure că exemplele construite de expertul în domeniu, au o oarecare consistență, spre exemplu: dacă s-a dat propoziția: "vreau să imi resetez parola din aplicația Saturn" etichetat cu entitatea: app: Saturn, să existe și contra exemple în care să se specifice și celălalt sens al cuvântului "Saturn", astfel modulul de NLU să poată identifica cu succes atunci când este vorba de aplicație sau despre planetă.

Abordarea folosită la gestionarea contextului ne oferă un cadru flexibil de a descrie obiectivele unei conversații. Este foarte ușor de urmărit procesul de decizie și se încadrează în minimul de robustețe cerut de industrie.


Modelele actuale cer un număr mare de date de antrenare, implementarea acestora în domenii specifice se lovește de lipsa exemplelor etichetate. Aceste neajunsuri fac ca atenția noastră să se îndrepte spre abordări ce privesc mai degrabă noi perspective legate de reprezentarea înțelesului.

Următorii pași în direcția de cercetare vor fi reprezentați de încercarea de a surprinde însemnătatea actelor de vorbire într-un dialog. Modul de succesiune al acestora dar și sensul semantic indus de ele ne vor fi de folos în procesul de învățare al politicilor de decizie ale agentului. Consider că acest demers ne va duce mai aproape de înțelegerea rațiunii și comportamentului din interacțiunile umane.
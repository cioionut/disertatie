\chapter{Introducere}

Privind comunicarea ca o nevoie de bază ne ajută sa vedem mai clar de ce procesarea limbajului natural este un element esențial in drumul nostru spre cunoaștere. Datorită actualului progres in acest domeniu ne putem bucura de ușurința cu care informațiile circulă între noi, făcând realizabil acest avânt tehnologic de care nu bucurăm cu toții.

\section{Motivația}

Luând contact tot mai des cu mediul de cercetare se contureaza tot mai bine gandul ca este nevoie de o întreaga armata de oameni de știință care să întoarca pe toate parțile un subiect pentru ca mai apoi cei mai de seamă dintre aceștia sa poată veni cu o sclipire. Fac această remarcă având în minte imaginea omului care obosit in a cauta raspuns la intrebarile existentiale, dar care se concentrează acum pe creație și descoperiri care sa facă viața mai ușoară și mai placută, lucru care dă naștere și mai multor mistere decat elucidări.

Dorința de a crea cu scopul de a aduce un aport de liniște în viața oamenilor este imboldul intrinsec ce ghidează acțiunile mele, așadar evaluând cunoștiințele mele, am decis sa imi aduc contribuția într-un domeniu atât important în drumul nostru spre o viziune clara precum un râu liniștit ce curge fară zgomot, dar limpede.


\section{Descrierea problemei}

Nu aș privi această chestiune precum o problemă, ci mai degrabă ca o nevoie. O nevoie ce survine în urma stilului nostru de viață dinamic și invelit în straturi de informație.

Cum limbajul natural este cel mai la indemâna instrument de comunicare, consider ca prin intermediul său vom putea satisface nevoia unei interfețe capabile să ușureze interacțiunea dintre noi și tehnologie.

\subsection{Privire de ansamblu}

Avem nevoie de un modul de NLU și un modul de DM pentru a pune bazele unui sistem de dialog

\section{Rezumatul capitolelor}

\begin{itemize}
	\item 
	Capitolul întâi vorbește in principal despre modul în care această lucrare își propune să rezolve nevoia de interacțiune cu tehnologia, dar și despre un capitol istoric privit prin ochii unei motivații îndraznețe.
	\item
	Atunci cand se rostește "progres" am în minte o  spirală a cunoștiintelor care se bazează unele pe altele. Precum această imagine implementarea acestei tehnologi de dialog impune un anumit progres precedent, așa că în capitolul doi vor fi prezentate aceste instrumente care fac posibilă aceasta tehnologie.
	\item
	In capitolul al treilea vor fi explicate modelele matematice care stau în spatele percepției de decizie.
	\item
	In partea a patra se prezintă modulul de înțelegerea limbajului natural
	\item
	Al cincelea capitol descrie modulul care ține contextul unei conversații.
	\item
	Iar in partea de final concluziile referitoare la studiul elaborat in această teză.
\end{itemize}

